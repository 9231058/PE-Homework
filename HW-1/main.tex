\documentclass[paper=a4, fontsize=11pt]{article}

%----------------------------------------------------------------------------------------
%	PACKAGES AND OTHER DOCUMENT CONFIGURATIONS
%----------------------------------------------------------------------------------------

\usepackage{amsmath,amsfonts,amsthm} % Math packages
\usepackage{sectsty} % Allows customizing section commands
\allsectionsfont{\centering \normalfont\scshape} % Make all sections centered, the default font and small caps

\usepackage{fancyhdr} % Custom headers and footers
\pagestyle{fancyplain} % Makes all pages in the document conform to the custom headers and footers
\fancyhead{} % No page header - if you want one, create it in the same way as the footers below
\fancyfoot[L]{} % Empty left footer
\fancyfoot[C]{} % Empty center footer
\fancyfoot[R]{\thepage} % Page numbering for right footer
\renewcommand{\headrulewidth}{0pt} % Remove header underlines
\renewcommand{\footrulewidth}{0pt} % Remove footer underlines
\setlength{\headheight}{13.6pt} % Customize the height of the header

\numberwithin{equation}{section} % Number equations within sections (i.e. 1.1, 1.2, 2.1, 2.2 instead of 1, 2, 3, 4)
\numberwithin{figure}{section} % Number figures within sections (i.e. 1.1, 1.2, 2.1, 2.2 instead of 1, 2, 3, 4)
\numberwithin{table}{section} % Number tables within sections (i.e. 1.1, 1.2, 2.1, 2.2 instead of 1, 2, 3, 4)

%\setlength\parindent{0pt} % Removes all indentation from paragraphs - comment this line for an assignment with lots of text
\setlength{\parskip}{1em}
\renewcommand{\baselinestretch}{1.5}

\usepackage{tikz-qtree}
\usepackage{lscape}

\usepackage{graphicx}
\graphicspath{ {images/} }

\usepackage{xepersian}
\settextfont[Path=fonts/]{Vazir.ttf}
%\setlatintextfont{Times New Roman}

%----------------------------------------------------------------------------------------
%	TITLE SECTION
%----------------------------------------------------------------------------------------

\newcommand{\horrule}[1]{\rule{\linewidth}{#1}} % Create horizontal rule command with 1 argument of height

\title{
\normalfont\normalsize
\includegraphics[scale=0.1]{aut}
\hspace{5cm}
\includegraphics[scale=0.1]{ceit} \\
\textsc دانشگاه صنعتی امیرکبیر \\
\textsc دانشکده مهندسی کامپیوتر و فناوری اطلاعات
\horrule{0.5pt} \\ [0.4cm] % Thin top horizontal rule
\huge ارزیابی کارآیی سیستم‌های و شبکه‌های کامپیوتری \\ % The assignment title
\huge تمرین اول \\ % The assignment title
\horrule{2pt} \\ [0.5cm] % Thick bottom horizontal rule
}

\author{پرهام الوانی}

\date{\normalsize\today} % Today's date or a custom date

\begin{document}

\maketitle % Print the title

\section{سوال اول}

\begin{align}
\begin{split}
    Prob\{X = n\} = \sum_{k=n}^{\infty} \frac{\eta^k e^{-\eta}}{k!} \binom{k}{n}p^n(1-p)^{k-n}
\end{split}
\end{align}

\section{سوال دوم}

\begin{align}
\begin{split}
    F(x) = \left \{
        \begin{tabular}{lr}
        \(x < 0\) & \(0\) \\
        \(0 < x < 30\) & \(\frac{60 + x}{90}\) \\
        \(x > 30\) & \(1\)
        \end{tabular}
      \right.
\end{split}
\end{align}

\section{سوال سوم}

\begin{align}
\begin{split}
    \int_{0}^{\infty} P(X > x)dx
    = \int_{0}^{\infty} (1 - F(x))dx
\end{split}
\end{align}

از روش جز به جز در انتگرال‌گیری استفاده می‌کنیم:

\begin{align}
\begin{split}
    \int_{0}^{\infty} (1 - F(x))dx
    = x(1 - F(x)) - \int_{0}^{\infty} -xf(x)dx
\end{split}
\end{align}

سمت راست این رابطه از دو قسمت تشکیل شده است،
قسمت انتگرالی برابر با امید ریاضی \lr{X}
می‌باشد و قسمت بدون انتگرال می‌بایست در دو نقطه صفر و بی‌نهایت
محاسبه گردد که خواهیم داشت:

\begin{align}
\begin{split}
    \int_{0}^{\infty} (1 - F(x))dx
    &= x(1 - F(x)) + E[x]\\
    &= 0 + E[x] = E[x]
\end{split}
\end{align}

\section{سوال چهارم}

\begin{align}
\begin{split}
    f_X(x) &= \int_{-\infty}^{\infty} \frac{1}{3}(x + y)dy \\
    &= \int_{0}^{2} \frac{1}{3}(x + y)dy = \frac{1}{3}(2x + 2)
\end{split}
\end{align}

\begin{align}
\begin{split}
    f_Y(y) &= \int_{-\infty}^{\infty} \frac{1}{3}(x + y)dx \\
    &= \int_{0}^{1} \frac{1}{3}(x + y)dx = \frac{1}{3}(y + \frac{1}{2})
\end{split}
\end{align}

\begin{align}
\begin{split}
    E[X] &= \int_{-\infty}^{\infty} \frac{1}{3}x(2x + 2)dx \\
    &= \int_{0}^{1} \frac{1}{3}(2x^2 + 2x)dx = \frac{5}{9}
\end{split}
\end{align}

\begin{align}
\begin{split}
    E[Y] &= \int_{-\infty}^{\infty} \frac{1}{3}y(y + \frac{1}{2})dy \\
    &= \int_{0}^{2} \frac{1}{3}(y^2 + \frac{y}{2})dy = \frac{11}{9}
\end{split}
\end{align}

\begin{align}
\begin{split}
    E[XY] &= \int_{0}^{1}\int_{0}^{2} \frac{1}{3}xy(x + y)dydx\\
    &= \int_{0}^{1} \frac{2x^2}{3} + \frac{8x}{9} dy\\
    &= \frac{6}{9}
\end{split}
\end{align}

\begin{align}
\begin{split}
    Cov[X,Y] = E[XY] - E[X]E[Y] = \frac{6}{9} - \frac{5}{9} * \frac{11}{9} = -\frac{1}{81}
\end{split}
\end{align}

\section{سوال پنجم}
ابتدا پارامتر‌های توزیع را بدست می‌آوریم.

\begin{align}
\begin{split}
    \alpha = \frac{E[x]^2}{Var[x]} = \frac{12 * 12}{48} = 3\\
    \beta = \frac{Var[X]}{E[x]} = \frac{48}{12} = 4\\
\end{split}
\end{align}

\begin{align}
\begin{split}
    Prob\{X > 12\} = 1 - Prob\{X \le 12\} =&\\
    1 - \sum_{i=0}^{2} \frac{(12/4)^i}{i!} e^{-12/4} = 1 - Poisson(\lambda = 3, X \le 2)
    &= 1 - 0.4232
\end{split}
\end{align}

\begin{align}
\begin{split}
    Prob\{X \le 6\} = \sum_{i=0}^{2} \frac{(6/4)^i}{i!} e^{-6/4} = Poisson(\lambda = 1.5, X \le 2) &=\\
    1/2 (Poisson(\lambda = 1.4, X \le 2) + Poisson(\lambda = 1.6, X \le 2)) &= 1/2(0.8335 + 0.7834)
\end{split}
\end{align}

\section{سوال ششم}

\section{سوال هفتم}

مکان شی بعد از \lr{n}
حرکت مجموع \lr{n}
متغیر تصادفی برنولی می‌باشد که
در صورت موفقیت امتیاز یک و در صورت شکست امتیاز منفی یک دارند.

\begin{align}
\begin{split}
    E[X_b] = 1 * P + -1 * (1 - P) =  1 + 2P\\
    Var[X_b] = E[X_b^2] - E[x_b]^2 = 1 - (1 + 2P)^2 = -4P - 4P^2 = 4P(1 + P)
\end{split}
\end{align}

\begin{align}
\begin{split}
    E[X_n] = n * (1 + 2P) \\
    Var[X_n] = n * 4P(1 + P)
\end{split}
\end{align}



\end{document}